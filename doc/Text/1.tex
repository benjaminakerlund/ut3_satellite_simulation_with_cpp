\section{Introduction}
\label{sec:cntroduction}

The aim of this practical project is to handle a radio telescope on-campus at \ac{UPS} and learn about antenna calibration and characterisation procedures. 
The instrument used is a radio telescope located on the roof of the \ac{OMP}. 
This is a $3\,m$ diameter steerable parabolic dish equipped with a $21\,cm$ wavelength receiver (frontend) and acquisition electronics (backend) capable of producing $1.2 \,MHz$ band spectra.

The measurement procedure consists of the student groups conducting measurements on the control computer located under the antenna, in room D022 on the \ac{IRAP} Belin site (\ac{OMP}). 
Observing time is around $45\,min$, during which students are asked to carry out an analysis, possibly including other archive data.
The aim is not to apply a pre-established protocol, but rather to carry out a research project with a few references.
All in all, the project is estimated to take around $5\,h$ to complete.
Sample observations and analysis scripts were provided.

The topic for our group is \textit{5. Estimate of hydrogen mass in the Galaxy from H1 line intensity} and includes the following work items:
\begin{enumerate}
    \item Calibrate antenna efficiency on the Sun 
    \item Point a direction in the Milky Way disc and interpret the spectrum.
    \item Convert total line intensity into hydrogen column density
    \item Estimate the mass of hydrogen in the Galaxy.
\end{enumerate}

