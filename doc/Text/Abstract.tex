\addcontentsline{toc}{section}{Abstract}
\section*{Abstract}



\begin{table}[h]
    \begin{tabular}{|p{\linewidth}|}
        \hline \\
        Using \ac{SPENVIS}, this report investigates the spacecraft environment for a specific pre-determined orbit corresponding to a so-called Molniya orbit.
        This report analyses the expected spacecraft environment and gives recommendations on design choices: specifically the thickness of solar panel cover glass, the thickness of aluminium shielding for memory devices and the selection of one Integrated Circuit chip over another. 
        \\~\\
        \textit{"\acs{SPENVIS} is \ac{ESA}'s \acf{SPENVIS}, a WWW interface to models of the space environment and its effects; including galactic cosmic rays, solar energetic particles, natural radiation belts, plasmas, gases, meteoroids and debris."} \cite{spenvis_webpage}.
        \\
        \hline
    \end{tabular}
    \label{abbreviations}
\end{table}
\textbf{Keywords:} SPENVIS, ESA, Spacecraft Environment, Space Environment, Single Event Effect, Single Event Upset, CMOS, Solar degradation, 

\vspace{2cm}
        \begin{figure}[h]
            \centering
            \includegraphics[width=0.3\textwidth]{Graphics/spenvislogo.png}
            \hspace{1cm}\includegraphics[width=0.3\textwidth]{Graphics/esa.png}
        \end{figure}