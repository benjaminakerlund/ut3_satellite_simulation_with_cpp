\begin{appendices}

\section{Logbook}
\label{sec:logbook}

\begin{table}[H]
\centering
    \begin{tabularx}{\textwidth} {
    | >{\hsize=0.45\hsize}X     % Date
    | >{\hsize=0.2\hsize}X     % Time
    | >{\hsize=0.4\hsize}X      % Activity
    | >{\hsize=1.95\hsize}X      % Remarks
    |}
    \hline
    \textbf{Date} & \textbf{Time} & \textbf{Activity} & \textbf{Remarks} \\ \hline\hline
        2025-10-02 & 2h & Lecture  & Introduction to course. C++ basics \\ \hline
        
        2025-10-30 & 2h & Lecture  &  Object Oriented Programming \\ \hline
        
        2025-11-20 & 2h & Lecture  & Pointers, References and Const \\ \hline
        
        2025-11-27 & 4h & Lecture  & Polymorphism \& Inheritance. Template Class and STL.  \\ \hline
        
        2025-12-11 & 4h & Lecture \& Exercise  & Lecture on Satellite Simulation. Began installing Ubuntu 16.04.07 with \textit{Oracle VM VirtualBox}. \\ \hline
        
        2025-12-18 & 4h & Exercise & Setup of Virtual Machine with whole class, problems in extracting the SimTG environment and setting this up on the VM. \\ \hline
        
        2025-01-08 & 4h & Exercise & Continuing to work with VM and SimTG, finally getting the environment to work and first example models and tests to run (\texttt{simple} and \texttt{multi}).  \\ \hline

        2025-01-15 & 4h & Exercise & Problems with assigning custom variables, arrays etc. to classes. Problems with \texttt{step()} function and testing how this works. SimTG installation broke on one computer, needed to reinstall everything. \\ \hline
        
        2025-01-20 & 4h & Exercise & Began working on model for solar cell, still some problems with creating new projects in SimTG. Decided to just use template project (\texttt{Add}) with template files/models (\texttt{simple} and \texttt{multi}) and develop our simulation directly in this project folder to avoid problems with importing libraries etc.     
        \\ \hline
        
        2025-01-29 & 4h & Exercise & Last session in class.  \\ \hline
        
        Dates?     && Independent work & \\ \hline

        2025-02-21 & 4h & Independent work & Finishing touches on Cell model + creating report file. \\ \hline

        2025-02-25 & 8h & Independent work & Creation of CSS model + work on report. Problems with building project in SimTG... Creating testfiles for Cell and CSS. \\ \hline
        
        2025-02-28 & 2h & Exam &  \\ \hline
    \end{tabularx}
    \begin{tabularx}{\textwidth} {|X|}
         \hline
         \textbf{Total hours logged: 36h scheduled + Xh independent work} \\ \hline
    \end{tabularx}

    
    \caption{Logbook of work throughout the course.}
    \label{tab:logbook}
    
\end{table}






\newpage
\section{Source code for Cell}
\begin{lstlisting}[frame=single,
numbers=left, basicstyle=\tiny, language = C++]

/*PROTECTED REGION ID(_bDafANNTEe-HHfwhf86eRQ_impl_cpp_before_includeimplementation) ENABLED START*/
#include <math.h>
#include <numeric>
/*PROTECTED REGION END*/

#include "Cell.hpp"

using namespace Add;

OBJECT_MAKER(Cell)

/*PROTECTED REGION ID(_bDafANNTEe-HHfwhf86eRQ_impl_cpp_after_includeimplementation) ENABLED START*/
//add user defined includes here
/*PROTECTED REGION END*/

void Cell::constructor() {
	/*PROTECTED REGION ID(_bDafANNTEe-HHfwhf86eRQdelegated_constructor) ENABLED START*/
	//add user defined code here
	/*PROTECTED REGION END*/

}
void Cell::destructor() {
	/*PROTECTED REGION ID(_bDafANNTEe-HHfwhf86eRQdelegated_destructor) ENABLED START*/
	//add user defined code here
	/*PROTECTED REGION END*/

}
void Cell::configure() throw (simtg::Exception) {
	/*PROTECTED REGION ID(_bDafANNTEe-HHfwhf86eRQdelegated_configure) ENABLED START*/
	//add user defined code here
	/*PROTECTED REGION END*/

}
void Cell::serializeExt(simtg::SerializationStream& stream_)
		throw (simtg::SerializationException) {
	/*PROTECTED REGION ID(_bDafANNTEe-HHfwhf86eRQserializeExt) ENABLED START*/
	//add user defined code here
	/*PROTECTED REGION END*/

}
void Cell::step() throw (simtg::Exception) {
	/*PROTECTED REGION ID(_bDafA9NTEe-HHfwhf86eRQ) ENABLED START*/
	float alpha = 22 * (M_PI / 180);		// sunsensor angle, converted to rad
	float I_max = 31 / 1000;			// max current in [A]
	float v = 9.6;						// largest incident coeff [-]
	float N_CSS = 2.3 * pow(10, -10)	// Noise coefficient
	float N = static_cast<float>(rand())
			/ (static_cast<float>(RAND_MAX / N_CSS)); // Noise

	float n[4][3];
	n[0] = {sin(alpha), cos(alpha), 0};		// +Y
	n[1] = {sin(alpha), -cos(alpha), 0};	// -Y
	n[2] = {sin(alpha), 0, cos(alpha)};		// +Z
	n[3] = {sin(alpha), 0, -cos(alpha)};	// -Z

	float dotProd = (_u_sun[0] * n[_axis][0]) +
					(_u_sun[1] * n[_axis][1]) +
					(_u_sun[2] * n[_axis][2]);

	float lambda = 1 - pow( ((2 / M_PI) * acos(dotProd)), v );

	_I_cell = ((I_max * (dotProd * lambda * _k * _e)) + N);
	/*PROTECTED REGION END*/
}

void Cell::init() throw (simtg::Exception) {

	/*PROTECTED REGION ID(_bDafANNTEe-HHfwhf86eRQ_startInit) ENABLED START*/
	// add user defined code here
	/*PROTECTED REGION END*/

	AsyncModelBase::init();

	/*PROTECTED REGION ID(_bDafANNTEe-HHfwhf86eRQ_init) ENABLED START*/
	//add user defined code here
	/*PROTECTED REGION END*/
}

/*PROTECTED REGION ID(_bDafANNTEe-HHfwhf86eRQ_impl_cpp_end_extensionsimplementation) ENABLED START*/
//add user defined includes here
/*PROTECTED REGION END*/

\end{lstlisting}


\newpage
\section{Example of test file for Single Photo Cell}
\label{sec:test_example}
The scenario is when the satellite is in eclipse and thus no current should be produced. Note for this test, we had to set the noise value to be 0 also.

\begin{lstlisting}[frame=single,
numbers=left, basicstyle=\tiny, language = C++]
    package Add;

import simtg.simops.base.SimopsException;

public class TestCell1 extends BaseTest {

	public static void main(String[] args) {
		new TestCell1().run();
	}

	protected void test() throws SimopsException {
		sim.createObject("Add", "Cell", "o");
		// sim.activateMethodCyclically("o.step",1);
		// sim.step();
		
		System.out.println("\n============ Test 1: zero all START ============");
		// PARAMETERS
		float test[] = { 0.0f, 0.0f, 0.0f };
		float k[] = {0.0f, 0.0f, 0.0f, 0.0f};
		int e = 0;	//0 if satellite in earth shadow...
		sim.writeFloatArray("o.In.u_sun", test);
		sim.writeFloatArray("o.In.k", k);
		sim.writeInt("o.In.e", e);
		sim.writeInt("o.In.axis", 0);
		//sim.timeStep(1.);
		// INPUTS DESC.
		sim.init();
		sim.step();
		System.out.println("Beginning test with the following inputs: \n" 
				+ "u_sun:\t{ 0, 0, 0 }\n" + "k:\t{0, 0, 0, 0}" + "\n" + "e:\t" + e);
		// TEST CASE
		if(sim.checkFloat("o.Out.I_cell", 0)) {System.out.println("Success!");} 
		else {System.out.println("Test failed...");}
		System.out.println("============ Test END ============\n");
		
		
		System.out.println("\n============ Test 2: real values for u, k and e = 0 START ============");
		// PARAMETERS
		float test2[] = { 1.0f, 0.0f, 0.0f };
		sim.writeFloatArray("o.In.u_sun", test2);
		sim.writeFloatArray("o.In.k", k);
		sim.writeInt("o.In.e", e);
		sim.writeInt("o.In.axis", 0);
		// INPUTS DESC.
		sim.init();
		sim.step();
		System.out.println("Beginning test with the following inputs: \n" 
				+ "u_sun:\t{ 1, 0, 0 }\n" + "k:\t{0, 0, 0, 0}" + "\n" + "e:\t" + e);
		// TEST CASE
		if(sim.checkFloat("o.Out.I_cell", 0)) {System.out.println("Success!");} 
		else {
			System.out.println("Test failed...");
			System.out.println(sim.readFloat("o.Out.I_cell"));
		}
		System.out.println("============ Test END ============\n");

		
		System.out.println("\n============ Test 3: real values for u and k, e = 0 START ============");
		// PARAMETERS
		float k3[] = { 1.0f, 1.0f, 1.0f, 1.0f };
		sim.writeFloatArray("o.In.u_sun", test2);
		sim.writeFloatArray("o.In.k", k3);
		sim.writeInt("o.In.e", e);
		sim.writeInt("o.In.axis", 0);
		sim.init();
		sim.step();
		sim.timeStep(1.);
		// INPUTS DESC.
		System.out.println("Beginning test with the following inputs: \n" 
				+ "u_sun:\t{ 1, 0, 0 }\n" + "k:\t{1, 1, 1, 1}" + "\n" + "e:\t" + e);
		// TEST CASE
		if(sim.checkFloat("o.Out.I_cell", 0)) {System.out.println("Success!");} 
		else {System.out.println("Test failed...");}
		System.out.println("============ Test END ============\n");
		
		System.out.println("\n============ Test 4: real and specific values for all START ============");
		// PARAMETERS
		int e4 = 1;
		sim.writeFloatArray("o.In.u_sun", test2);
		sim.writeFloatArray("o.In.k", k3);
		sim.writeInt("o.In.e", e4);
		sim.writeInt("o.In.axis", 0);
		sim.init();
		sim.step();
		sim.timeStep(1.);
		// INPUTS DESC.
		System.out.println("Beginning test with the following inputs: \n" 
				+ "u_sun:\t{ 1, 0, 0 }\n" + "k:\t{1, 1, 1, 1}" + "\n" + "e:\t" + e4);
		// TEST CASE
		if(sim.checkFloat("o.Out.I_cell", 0.010825224571338f)) {System.out.println("Success!");} 
		else {System.out.println("Test failed...");}
		System.out.println("============ Test END ============\n");
		
		
		System.out.println("\n============ Test 5: specific values for all START ============");
		// PARAMETERS
		float test5[] = {0.0f, 1.0f, 1.0f};
		sim.writeFloatArray("o.In.u_sun", test5);
		sim.writeFloatArray("o.In.k", k3);
		sim.writeInt("o.In.e", e4);
		sim.writeInt("o.In.axis", 0);
		sim.init();
		sim.step();
		sim.timeStep(1.);
		// INPUTS DESC.
		System.out.println("Beginning test with the following inputs: \n" 
				+ "u_sun:\t{ 0, 1, 1 }\n" + "k:\t{1, 1, 1, 1}" + "\n" + "e:\t" + e4);
		// TEST CASE
		if(sim.checkFloat("o.Out.I_cell", 0.028742661027327f)) {System.out.println("Success!");} 
		else {System.out.println("Test failed...");}
		System.out.println("============ Test END ============\n");
	}
}

	}
\end{lstlisting}

\end{appendices}
